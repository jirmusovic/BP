% Tento soubor nahraďte vlastním souborem s obsahem práce.
%=========================================================================
% Autoři: Michal Bidlo, Bohuslav Křena, Jaroslav Dytrych, Petr Veigend a Adam Herout 2019

% Pro kompilaci po částech (viz projekt.tex), nutno odkomentovat a upravit
%\documentclass[../projekt.tex]{subfiles}

%\newenvironment{definice}{\begin{quote}\textbf{Definice:}}{\end{quote}}

\newtheorem{definition}{Definice}
\newtheorem{remark}{Pozn\'amka}
\newtheorem{example}{Příklad}
\newtheorem{graph}{Obr\' azek}
\newtheorem{sentence}{Věta}
\newtheorem{tabul}{Tabulka}
\newtheorem{corollary}{D\r usledek}
\newcommand{\mycomment}[1]{}
%\newcommand{\comment}[1]{}

\chapter{Úvod}
%\comment{
%\color{blue} Pár vět, které pak chci zakomponovat do \'uvodu nějak:
%                \\ 
%                \color{black} Zatímco bivalentní logika pracuje s binárními hodnotami "pravda" a %"nepravda", fuzzy logika umožňuje pracovat s hodnotami, které se pohybují mezi těmito extrémy.

%}




\chapter {Fuzzy mno\v ziny a fuzzy logika}
\section{Fuzzy mno\v ziny} 

%\comment{
%V klasické i fuzzy logice se často setkáváme se základními pojmy množina a prvek množiny. Množina se obecně %vysvětluje jako souhrn, soubor nebo skupina objektů. Tyto objekty pak nazýváme prvky dané množiny. Nejvíce %charakteristická vlastnost množin je, že je jednoznačně určena svými prvky, ale ignoruje jejich pořadí a %další jejich struktury. Množina, která neobsahuje žádné prvky, se nazývá prázdná množina.
%}

V běžném životě se často setkáváme s nepřesnými pojmy, jako jsou např. \clqq málo\crqq, \clqq hodně\crqq \space či třeba \clqq trochu\crqq. Jak ale takovou vágnost vyjádřit v matematice? Pokud člověk prohlásí tvrzení, že je \clqq celkem mladý\crqq, znamená to, že je mladý nebo ne? K zápisu a práci s takovými výroky se pak ve fuzzy logice využívají fuzzy množiny a operace s nimi.

Nech\v t je např. vypsáno výběrové řízení modelingové agentury s požadavkem, že hledají vysoké uchazeče. Taková informace je tedy poněkud vágní, ačkoliv v běžném životě lehce srozumitelná. Pokud by se někdo pokusil definovat pojem \clqq vysoký člověk\crqq \space pomocí ostré množiny, musel by nejprve stanovit hranici, kdy je člověk vysoký, např. 180 centimetr\r u, množina všech vysokých lidí V by měla charakteristickou funkci:

    $$X_V:(x)=\begin{cases} 1, & \mbox{pokud }  x \geq 180,\\    0, & \mbox{pokud } x < 180,  \end{cases}$$

    čímž by došlo k paradoxu, že člověk který má 179,9 centimetr\r u je považován za nízkého a člověk se 180 centimetry za vysokého.

    Takový problém je zp\r usoben tím, že byla výška modelována jako vlastnost, kterou m\r uže mít člověk pouze v nulové míře nebo na 100 \%. Za určitě vysokého člověka byl považován někdo se 180 centimetry, 175 centimetr\r u vysoký člověk byl podle stejného přístupu určitě n\'izk\'y. Přičemž 175 centimetr\r u vysoký člověk je běžně považován do určité míry za vysokého. Přesně z toho d\r uvodu se zavedl pojem \textit{fuzzy množina}, která je podmno\v zinou tzv. \textit{univerza}, které p\v redstavuje základní prostor. 
    
    \begin{definition}
    \cite{navara}
        Fuzzy podmnožina univerza X (stručně fuzzy množina) je objekt A, který popisuje (zobecněná) charakteristická funkce, která se nazývá funkce příslušnosti $\mu: X \rightarrow [0,1]$. 
    \end{definition}
    
    Fuzzy množina nebo také neostrá množina je pak množina prvků takových, jejichž afiliace je k této množině odstupňovaná. Fuzzy množina staví na stejných pravidlech, jako klasická množina, s tím rozdílem, že příslušnost nemusí nabývat jen hodnot 0 a 1, ale jakoukoliv hodnotu z intervalu [0,1]. 
   
    Funkce příslušnosti umož\v nuje vyjádřit částečnou příslušnost k množinám na intervalu [0,1], tedy jak moc lze označit pojem za \clqq pravdu\crqq \space či \clqq nepravdu\crqq. Díky čemuž se pak dají matematicky vyjádřit vágní pojmy jako \clqq docela dost\crqq, \clqq málo\crqq \space nebo \clqq mnoho\crqq \space apod.

     Pokud by se v předchozím zmíněném výběru modelingové agentury označila výška 180 centimetr\r u stupněm vlastnosti \clqq vysoký\crqq \space 1, pak je možné přiřadit výšce 175 cemtimetr\r u například stupe\v n 0,85. Když je každému prvku $x$ ze základního prostoru výšky [0, $250$[ centimetr\r u přiřazeno číslo z intervalu [0,1], které vyjadřuje míru, ve které je člověk mající výšku $x$ vysoký, lze pak získat funkci, která kompletně charakterizuje pojem vysoký člověk: $\mu_V: X \to [0,1]$, kde $X = [0, 250[$ vyjadřuje základní prostor, neboli univerzum. Takovou funkcí je např. funkce $\mu_V:  [0, 250[ \rightarrow [0,1]$, 

    $$\mu_V(x)=\begin{cases} 1, & \mbox{pokud }  180\leq x \leq 250,\\ 
    \frac{x}{20} - 8, & \mbox{pokud } 160 \leq x < 180,\\
    0, & \mbox{pokud } x < 160,  \end{cases}$$

    jíž graf je vykreslen níže.

    \begin{graph} Graf funkce příslušnosti fuzzy množiny V - \clqq Vysoký člověk\crqq.\\
    \begin{figure}[h]
        \includegraphics[scale=0.65]{template-fig/vysoky_clovek.pdf}
        \centering
    \end{figure}
\end{graph}     



Pokud bychom chtěli modelovat funkci např. pojmu \clqq asi 3\crqq, lze zvolit několik možných řešení tohoto problému. Každý pak vyjadřuje jiný rozptyl možných řešení.
\begin{graph} Grafické znázornění funkcí příslušnosti výroku P \clqq asi 3\crqq.\\
    \begin{figure}[h]
        \includegraphics[scale=0.65]{template-fig/asi_3.pdf}
        \centering
    \end{figure}

\end{graph}


\section{Fuzzy logick\'e spojky}

Fuzzy logické spojky n\'am umo\v z\v nují vyjádřit vágnost a neurčitost v logických operacích. Tyto spojky se často využívají v aplikacích jako jsou řízení průmyslových procesů, rozpoznávání vzorů, vývoj umělé inteligence a dalších oblastech, kde je potřeba pracovat s neurčitými informacemi.\\

V\v sechny fuzzy logick\'e spojky jsou monot\'onn\'im roz\v s\'i\v ren\'im klasick\'ych logick\'ych spojek. V dalších kapitolách budou postupně představeny základní fuzzy spojky, mezi kter\'e  patří 
fuzzy negace, konjunkce, disjunkcie a implikace. 

\subsection{Fuzzy negace}

Fuzzy negace představují klíčový koncept v oblasti fuzzy logiky. Tyto negace jsou zobecn\v en\'im klasick\'ych negac\'i. Základní vlastnosti fuzzy negací jsou:

\begin{enumerate}
\item \textbf{Soudržnost:}\\
Pro hodnoty $0$ a $1$ se fuzzy negace shoduje s klasickou negac\'i.
    \item \textbf{Kontinuita hodnot:} \\
        Fuzzy negace nepracuje jenom s  ostrými hodnotami „pravda“ a „nepravda“, ale na stupni neurčitosti či specifickým číselném zápisu, který reflektuje stupeň nepravdivosti.
    \item \textbf{Monotonie:} \\
        Fuzzy negace jsou monot\'onn\'im roz\v s\'i\v ren\'im klasick\'e negace, což znamená, že s rostoucími vstupn\'imi hodnotami klesaj\'i hodnoty výstupn\'e.
    
\end{enumerate}

\begin{definition}
\cite{Kolo} Fuzzy negací nazýváme každou funkci $N:[0, 1] \to [0, 1]$  s vlastnostmi:
    \begin{enumerate}
        \item  $N(0) = 1, N(1) = 0,$
        \item $\forall  x, y \in [0, 1]: x < y \Rightarrow{} N(x) \geq N(y).$
     \end{enumerate}
\end{definition}
     \begin{example}
         Funkce $N_S(x)=1-x$ spl\v nuje vlastnosti fuzzy negace na intervalu $[0,1].$ Je to nejzn\'am\v ej\v s\'i fuzzy negace, kterou ve sv\'e pr\'aci p\v redstavil Lotfi A. Zadeh a obvykle je ozna\v cov\'ana jako standardn\'i negace.   Fuzzy negace nemus\'i b\'yt nutn\v e spojit\'a funkce. Zn\'am\'e p\v r\'iklady nespojit\'ych fuzzy negac\'i jsou n\'asleduj\'ic\'i funkce:
         $$ N_{\bot}(x)=\begin{cases} 1, & \mbox{pokud }x=0, \\
         0, & \mbox{pokud }x\in \mbox{]0, 1]}, \end{cases} \mbox{  }
         N_{\top}(x)=\begin{cases} 1, & \mbox{pokud }x\in \mbox{[0, 1[}, \\
         0, & \mbox{pokud }x=1. \end{cases}$$
         Funkce $N_{\bot}$ je nejmen\v s\'i a funkce $N_{\top}$ je nejv\v et\v s\'i fuzzy negace, proto
         \begin{equation}
        N_{\bot}(x) \geq N(x) \geq N_\top(x)  \mbox{ pro každé } x \in [0, 1]. \
    \end{equation}
    V literatu\v re jsou negace $N_\bot, N_\top$ zn\'am\'e jako Gödelovy fuzzy negace.
    
    Funkce $N_1(x) = \sqrt{1-x}$ a $N_2(x) = \sqrt{1-x^2}$ jsou zřejmě pro $ x \in [0,1]$ nerostoucí a tak\'e plat\'i:
        \begin{enumerate}
            \item $N_1(0) = \sqrt{1-0} = 1, 
                    N_1(1) = \sqrt{1-1} = 0, $
            \item $N_2(0) = \sqrt{1-0^2} = 1,
                    N_2(1) = \sqrt{1-1^2} = 0.$
        \end{enumerate}
         Tedy i tyto funkce jsou fuzzy negace.

         \begin{graph} Grafy dříve zmíněných funkcí $N_s$, $N_{\bot}$,$ N_{\top}$, $N_1$ a $N_2.$\\
         
            \begin{figure}[h]
                %\hspace{-0.1cm}
                \includegraphics[scale=0.56]{template-fig/negace.pdf}
                \centering
            \end{figure}
        \end{graph}
    \end{example}

   N\v ekter\'e fuzzy negace maj\'i dal\v s\'i zaj\'imav\'e vlastnosti. 

    \begin{definition}
    \cite{Kolo}
        Kdy\v z je $N: [0,1] \to [0,1]$ klesající a spojitá fuzzy negace, nazývá se striktní negace.
        Striktn\'i fuzzy negace, kter\'a je involutivní, tedy, pro kterou plat\'i $N(N(x)) = x $ pro každé $ x \in [0,1]$, se nazývá silná fuzzy negace.
    \end{definition}

    \begin{example}
        Spojit\'e neg\'atory $N_S, N_1, N_2$ z p\v redchoz\'iho p\v rikladu jsou evidetn\v e striktn\'i, proto\v ze jsou ryze monot\'onn\'i. D\'ale plat\'i:
        $$N_S(N_S(x))=1-(1-x)=x, \mbox{pro ka\v zd\'e } x \in [0,1],$$
                $$N_1(N_1)) = N_1(\sqrt{1-x}) = \sqrt{1-\sqrt{1-x}} \neq x, \mbox{ pro t\'em\v e\v r každé } x \in ]0,1[,$$ $$N_2(N_2) = N_2(\sqrt{1-x^2}) = \sqrt{1-(\sqrt{1-x^2})^2} = x,
                \mbox{pro každé } x \in [0,1].$$
Takže negace $N_S$ a $N_2$ jsou involutivní a tím pádem tak\'e siln\'e negace, naopak
            negace $N_1$ není involutivní, tedy ani siln\'a fuzzy negace.
                
            \end{example}

    \begin{remark}
        Z rovnosti $N(N(x))=x$ plyne pro bijektivn\'i funkce tak\'e rovnost $N(x)= N^{-1}(x).$ 
         Grafick\'a interpretace  t\'eto rovnosti  pro fuzzy negace je taková, že grafy involutivních fuzzy negací jsou osově souměrné podle osy $y = x$.
        \begin{graph} Ukázka osové souměrnosti bijektivní funkce f(x) = $\sqrt{1-x^2}$ podle osy y = x pro $x \in [0,1]$.\\
            
            \begin{figure}[h]
                \hspace{-1cm}
                \includegraphics[scale=0.65]{template-fig/soumernost.pdf}
                \centering
            \end{figure}
        \end{graph}
    \end{remark}


    
\subsection{Fuzzy konjunkce}

Fuzzy konkjunkce se využívají pro spojení dvou a více podmínek s ohledem na jejich vágnost a neostrost. Místo výsledného \clqq ano\crqq \space nebo \clqq ne\crqq \space je použita hodnota  $x \in [0,1]$, která reprezentuje, do jaké míry jsou podmínky splněny.  
Základní vlastnosti fuzzy konjunkcí jsou:
\begin{enumerate}
    \item \textbf{Soudržnost:}\\
    Pro hodnoty 0 a 1 se fuzzy konjunkce shoduje s klasickou konjunkcí.
    \item \textbf{Kontinuita hodnot:}\\
    Fuzzy konjunkce nepracuje jenom s  ostrými hodnotami „pravda“ a „nepravda“, ale na stupni neurčitosti či specifickým číselném zápisu, který reflektuje stupeň pravdivosti.
    \item \textbf{Monotonie:}\\
    Fuzzy konjunkce je monot\'onní operace, což znamená, že s rostoucími vstupními hodnotami rostou i výsledky jejich konjunkce.
\end{enumerate}
\begin{definition}
    \cite{KMP}
    Neklesající zobrazení C: $[0,1]^2 \rightarrow [0,1]$ se nazývá konjunktor, pokud pro libovolné a,b $\in$ [0,1] platí
    C(a,b) = 0 pokud a = 0, nebo b = 0,
    C(1,1) = 1.
    \end{definition}

\begin{remark}
    Fuzzy konjunkce mohou být modelovány pomocí triangulárních norem, kterým je věnována celá následující podkapitola.
\end{remark}

\subsection{Triangul\'arn\'i normy} 

Jak již bylo zmíněno, fuzzy konjunkce mohou být modelov\' any pomocí triangulárních norem, zjednodušeně t-norem. 
\begin{definition}
\cite{KMP}
    Triangulární norma je binární operace na jednotkovém intervalu [0,1], t.j. funkce $T: [0,1]^2 \rightarrow [0,1]$ taková, že pro každé $x, y, z \in [0,1]$ jsou splněné následující axiomy:
    \begin{enumerate}
        \item \textbf{Komutativnost:} $T(x,y) = T(y,x)$,
        \item \textbf{Asociativita:} $T(x, T(y, z)) = T(T(x, y), z)$,
        \item \textbf{Monotónnost:} pokud $y \geq z$ pak $T(x, y) \geq T(x, z)$,
        \item \textbf{Okrajová podmínka:} $T(x, 1) = x$.
    \end{enumerate}
\end{definition}

\begin{example} N\'asleduj\'ic\'i funkce vždy spl\v nují 3 ze 4  axiom\r u, což ukazuje, že jsou na sobě axiomy nez\' avislé, a tedy funkce není t-normou, pokud nespl\v nuje všechny čtyři axiomy.\\ 
    Funkce $F_1 : [0, 1]^2 \to [0, 1]$ daná přepisem
    $$F_1(x,y) = x,$$
    spl\v nuje axiomy 2., 3., 4., ale nespl\v nuje axiom 1.
    
    Funkce $F_2 : [0, 1]^2 \to [0, 1]$ daná přepisem
    $$F_2(x,y) = x.y.max(x,y),$$
    spl\v nuje axiomy 1., 3., 4., ale nespl\v nuje axiom 2.

    Funkce $F_3 : [0, 1]^2 \to [0, 1]$ daná přepisem
    $$ F_3(x,y)=\begin{cases} 0.5, & \mbox{pokud }(x, y) \in ]0,1[^2, \\
         min(x,y), & \mbox{jinak,}\end{cases} $$
    spl\v nuje axiomy 1., 2., 4., ale nespl\v nuje axiom 3.
    
    Funkce $F_4 : [0, 1]^2 \to [0, 1]$ daná přepisem
    $$F_4(x,y) = 1,$$
    spl\v nuje axiomy 1., 2., 3., ale nespl\v nuje axiom 4.

  \end{example}

Mezi základní triangulární normy se řadí minimová, součinová, Łukasiewiczova t-norma a drastický součin.
\begin{example}
\cite{KMP}
    \begin{enumerate}
    \item \textbf{Minimová t-norma} $T_M: [0,1]^2 \rightarrow [0,1]$
    $$T_M(x,y) = min(x,y).$$
    \item \textbf{Součinová t-norma} $T_P: [0,1]^2 \rightarrow [0,1]$
    $$T_P(x,y) = x.y.$$
    \item \textbf{Łukasiewiczova t-norma} $T_L: [0,1]^2 \rightarrow [0,1]$
    $$T_L(x,y) = max(x+y-1,0).$$
    \item \textbf{Drastický součin} $T_D: [0,1]^2 \rightarrow [0,1]$
    $$T_D:(x)=\begin{cases} min(x,y), & \mbox{pokud }  max(x,y) = 1,\\ 
    0, &  jinak.  \end{cases}$$\\
\end{enumerate}
\end{example}


\begin{graph} Minimová t-norma $T_M$, Součinová t-norma $T_P$, Łukasiewiczova t-norma $T_L$, Drastický součin $T_D$.
   \begin{figure}[H]
    \hspace{-1cm}
        \includegraphics[scale=0.7]{template-fig/t_normy.pdf}
        \centering
    \end{figure}
\end{graph}

\begin{remark}
    Je zřejmé, že pro každou t-normu T platí
    $$T(1,x)=T(x,1)=x,$$
    $$T(0,x)=T(x,0)=0.$$
\end{remark}
\begin{definition}
\cite{KMP}
    \begin{itemize}
        \item Pokud pro t-normy $T_1$ a $T_2$ je
        pro každý bod $(x,y) \in [0,1]^2$ splněná nerovnost\\
        $T_1(x,y)\leq ~T_2(x,y),$ hovo\v ríme, že $T_1$ je slabší než $T_2$,
        nebo $T_2$ je silnější než $T_1$ a píšeme $T_1\leq T_2$.
        \item  Pokud pro t-normy $T_1$ a $T_2$ platí, že $T_1 \leq T_2$ a
        $T_1 \ne T_2,$ t.j. pokud $T_1 \leq T_2$, ale $T_1(x_0,y_0) <
        T_2(x_0,y_0)$ pro nějaký bod $(x_0,y_0) \in [0,1]^2$, tak $T_1<T_2$.
    \end{itemize}
\end{definition}


Existuje nekone\v cn\v e mnoho t-norem, dokonce pak i cel\'e t\v r\'idy parametrick\' ych  t-norem. Mezi nejzn\' am\v ej\v s\' i parametrické třídy patří:
\begin{itemize}
    \item \textbf{Frankovy t-normy:}
    $$T_p^F:(x,y)=\begin{cases} T_M(x,y), & \mbox{pokud }  p = 0,\\ 
                                T_P(x,y), & \mbox{pokud } p = 1,\\
                                T_L(x,y), & \mbox{pokud } p = +\infty,\\
                                log_p(1+\frac{(p^x-1)(p^y-1)}{p-1}), & \mbox{pokud } jinak, 
                                \end{cases}$$
    \item \textbf{Schwarz-Skalarovy t-normy:}
    $$T_p^{SS}:(x,y)=\begin{cases} T_M(x,y), & \mbox{pokud }  p = -\infty,\\ 
                                (x^p+y^p-1)^\frac{1}{p}, & \mbox{pokud }  -\infty < p < 0,\\ 
                                T_P(x.y), & \mbox{pokud } p = 0,\\
                                T_D(x,y), & \mbox{pokud } p = +\infty,\\
                                (max(0, x^p+y^p-1))^\frac{1}{p}, & \mbox{pokud } 0 < p < +\infty, \end{cases}$$
    \item \textbf{Yagerovy t-normy:}
    $$T_p^Y:(x,y)=\begin{cases}  T_D(x,y), & \mbox{pokud } p = 0,\\
                                max(0,1-((1-x)^p+(1-y)^p)^\frac{1}{p} , & \mbox{pokud } 0 < p < +\infty,\\
                                T_M(x,y), & \mbox{pokud } p = +\infty,
                                \end{cases}$$
    \item \textbf{Sugeno-Weberovy t-normy:}
    $$T_p^{SW}:(x,y)=\begin{cases}  T_D(x,y), & \mbox{pokud } p = -1,\\
                                    max(0,\frac{x+y-1+pxy}{1+p}) , & \mbox{pokud } -1 < p < +\infty,\\
                                    T_P(x,y), & \mbox{pokud } p = +\infty.
                                    \end{cases}$$
\end{itemize}

\subsection{Algebraick\'e vlastnosti t-norem}

Každou \' usečku délky $b$ lze pokrýt nějakým počtem \' useček délky $a > 0.$ Tato věta lze vyjádřit i jinak: pro dvě libovolná kladná a reálná čísla $a,b$ existuje přirozené číslo $n$ takové, že $a*n>b,$ což je archimedovská vlastnost reálné osy vzhledem ke sčítání. Velmi podobně je definována i archimedovská vlastnost pro násobení. Pro t-normy je pak archimedovská vlastnost definována následovně:
\begin{definition}
\cite{KMP}
    T-norma je archimedovská, pokud pro všechny body $(x,y) \in ]0,1[^2$ existuje $n \in N$ takové, že $$x_T^{(n)} < y,$$
    p\v ri\v cem\v z $x^{(n)}_T=$
\end{definition}
Ověřit archimedovskou vlastnost lze následovným zp\r usobem:
\begin{sentence} \cite{KMP}
    Triangulární norma T je archimedovská právě, když pro každé $(x,y) \in ]0,1[^2$ je $$\lim_{n \to \infty}x_T^{(n)} = 0.$$
\end{sentence}
\begin{sentence} \cite{KMP}
    Pokud je t-norma T archimedovská, tak pro každé $x \in ]0,1[$ platí $T(x,x) < x.$\\
    Pokud je t-norma spojitá zprava, pak je archimedovská právě, když pro každé $x \in ]0,1[$ platí $T(x,x) < x.$
\end{sentence}

    Díky p\v redchozí v\v et\v e lze  archimedovskou vlastnost t-norem pro zprava spojit\'e t-normy také definovat pomocí diagonální nerovnosti $$T(x,x) < x, \mbox{ pro } x \in ]0,1[.$$

    Nejd\r uležitější algebraickou vlastností funkcí je monot\' onnost. U t-norem se pak rozlišuje několik r\r uzných typ\r u monot\' onnosti:
    \begin{definition}
    \cite{KMP}\\
        \begin{itemize}
            \item T-norma $T$ je {\em striktně
            monot\' onní}, pokud
            je rostoucí na $]0,1]^2$ jak funkce $ T:[0,1]^2 \rightarrow [0,1]$ anebo
            ekvivalentně,
            $$ \text {pokud} \hskip 3mm x \in \hskip 1mm ]0,1] \hskip 3mm \text{a} \hskip 3mm y < z, \hskip 3mm \text {tak} \hskip 3mm T(x,y) < T(x,z). $$
            \item  T-norma $T$ je {\em sdruženě striktně
            monot\' onní}, pokud platí:
            $$ \text{pokud} \hskip 3mm  x < x'\hskip 3mm \text{a} \hskip 3mm y<y',
            \hskip 3mm  \text{tak} \hskip 3mm   T(x,y)<T(x',y').$$
            \end{itemize}
    \end{definition}

    Dal\v s\'i d\r uležité  pojmy jsou \textit{dělitel nuly a nilpotentn\'i prvek}:
    \begin{definition}
        \cite{KMP}\\
        Prvek $x \in ]0,1[$ lze nazvat {\em dělitelem nuly} dané t-normy $T$, pokud
        existuje $y \in ]0,1[$ takové, že $T(x,y) = 0.$
        Prvek $x \in ]0,1[$ je {\em nilpotentním prvkem dané t-normy $T$}, pokud existuje $n \in N$ takové,
        že $ x_T^{(n)} =0.$
    \end{definition}

    \begin{example}
        Je zřejmé, že výše zmíněné t-normy $T_P$ a $T_M$ nemají ani nilpotentní prvek, ani dělitele nuly. Naopak pro t-normy $T_D$ a $T_L$ platí, že každé $x \in ]0,1[$ je dělitelem nuly a i nilpotentním prvkem těchto t-norem.
    \end{example}

    \begin{definition}
    \cite{KMP}
        Triangul\'arn\'i norma $T$ se naz\'yv\'a {\em nilpotentní}, pokud je spojitá a každé $x
        \in ]0,1[$ je jejím nilpotentním prvkem.
    \end{definition}    
    \begin{definition}
    \cite{KMP}
        Triangul\'arn\'i norma $T$ se naz\'yv\'a {\em striktní}, pokud je
            spojitá a striktně monot\' onní.
    \end{definition} 
    
     Vztah nilpotentních a striktních t-norem, které jsou archimedovské a spojité, lze pak vyjádřit následovně:
    \begin{sentence}\cite{KMP}
        Nech\' t je T spojitá archimedovská t-norma. Potom jsou následující tvrzení ekvivalentní:
        \begin{itemize}
        \item T je nilpotentní.
        \item  Existuje alespo\v n jeden nilpotentní prvek dané t-normy T.
        \item T není striktní.
        \item  T má dělitele nuly.
        \end{itemize}
    \end{sentence}
    Mezi nejtypičtější příklady nilpotentních t-norem patří Łukasiewiczova t-norma a všechny archimedovsk\'e, nilpotentní t-normy jsou s ní izomorfní. Naopak, v\v sechny archimedovsk\'e, striktn\'i t-normy jsou izomorfn\'i se sou\v cinovou t-normou $T_P.$
    

\subsection{Konstrukce triangulárních norem}

V t\'eto pr\'aci se v\v enuji tak\'e konstrukc\'im t-norem. Triangul\'arn\'i normy lze konstruovat pomoc\'i funkc\'i jedn\'e prom\v enn\'e, ale tak\'e z ji\v z existuj\'ic\'ich t-norem. Mezi nejznámější konstrukce t-norem patří:
\begin{itemize}
    \item \textbf{Ordinální součet:}
    \cite{KMP}
        Nech\v t $(T_\alpha)_{\alpha \in A}$ je třída t-norem a
        nech\v t $(]a_\alpha,e_\alpha[)_{\alpha \in A}$ je třída po dvojcích
        disjunktních otevřených podinterval\r u intervalu $[0,1].$ Potom funkce
        \hbox{$T:[0,1]^2\rightarrow [0,1]$} daná předpisem
        $$T(x,y)=\begin{cases} a_\alpha + (e_\alpha - a_\alpha ).T_\alpha (\frac
        {x-a_\alpha}{e_\alpha - a_\alpha}, \frac {y-a_\alpha}{e_\alpha - a_\alpha}),
        &\mbox {pokud $(x,y) \in ]a_\alpha ,e_\alpha [^2$,}
        \\\min(x,y), &\mbox {jinak,} \end{cases}$$
        je t-normou, kterou nazývame {\em ordinálním součtem} sčítanc\r u $\langle a_\alpha ,e_\alpha
        ,T_\alpha \rangle,$ \mbox{$ \alpha \in A.$}

    \item \textbf{Aditivní a multiplikativní generování:}
        \cite{KMP}
        Nech\v t je funkce $f:[0,1] \to [0,\infty]$ spojitá a klesající, přičemž
        $f(1)=0$, potom je předpisem
        $$ T_{<f>}(x,y)=f^{-1}(\min(f(x)+f(y),f(0)))$$
        dána t-norma a funkce $f$ se nazývá aditivní generátor t-normy
        $T_{<f>}.$
        
        Nech\v t je funkce $g:[0,1] \to [0,1]$ spojitá a rostoucí, přičemž
        $g(1)=1$, potom je předpisem
        $$ T^{<g>}(x,y)=g^{-1}(\max(g(x).g(y),g(0)))$$
        dána t-norma a funkce $g$ se nazývá multiplikativní generátor t-normy
        $T^{<g>}.$
    \item \textbf{$\varphi$-transformace}
        \cite{KMP}
        Pokud je $\varphi$ rostoucí bijekce uzavřeného jednotkového intervalu, potom
        předpisem
        $$T_\varphi (x,y)=\varphi ^{-1}(T(\varphi(x),\varphi(y)))\space \text {pro $(x,y)
        \in [0,1]$}$$
        je dána t-norma, kterou nazýváme $\varphi-${\em transformací t-normy}
        $T.$

\end{itemize}

\begin{remark}
    Triangul\'arn\'i norma $T$ je striktní právě tehdy, když pro její aditivní generátor platí $f(0) = +\infty$. Nilpotentní je v momentě, když pro  její aditivní generátor platí $f(0) < +\infty$.
\end{remark}
\subsection{Pseudoinverzn\'i funkce}

Krom\v e   spojit\'ych t-norem existuj\'i tak\'e t-normy nespojit\'e a dokonce se daj\'i generovat. Jak je tedy možné konstruovat nespojité t-normy? Například funkce $t: [0,1] \rightarrow [1,2],$ daná předpisem $$t(x)= \begin{cases} 2-x, & \mbox {pokud }x \in [0,1[,
    \\ 0, & \mbox {pokud} x = 1,
    \end{cases}$$
    
    je aditivním generátorem drastického součinu, což zřejmě znamená, že generátory nespojitých t-norem nemusejí být nutně bijekce. Z toho d\r uvodu je potřeba zobecnění inverzní funkce.

      Pro neklesající  funkce je definice zobecn\v en\'e inverzn\'i funkce n\'asledovn\'i:
\begin{definition}
    \cite{hlinena}
    Nech\v t je $f:[a,b] \rightarrow [c,d]$ neklesající funkce, pak pro každé $y \in [c,d]$ je předpisem $$f^{(-1)}(y) = sup(x \in [a,b];f(x)<y)$$
    definována pseudoinverzní funkce $f^{(-1)}$ k dané funkci f.
\end{definition}

Pro lep\v s\'i pochopen\'i t\'eto definice, uv\'ad\'ime ilustra\v cn\'i p\v r\'iklady:

\begin{example} 
    Nech\v t je  $f_1:[0,1] \rightarrow [0,1]$  dána předpisem:
    $$f_1(x)=x^2.$$
   Pro bijektivn\'i funkce jsou inverzn\'i a pseudoinverzn\'i funkce toto\v zn\'e: \newline$f_1^{(-1)}:[0,1] \rightarrow [0,1]$ zřejmě bude platit:
    $$f_1^{(-1)}(x)=f_1^{-1}(x)= \sqrt{x}.$$
    Proto složením funkcí $f_1$ a $f_1^{(-1)}$ je
    identita na intervale $[0,1].$
\end{example} 

    \begin{graph} Ukázka složení funkcí $f_1^{(-1)} $ a $ f_1$, nebo také zevšeobecnění inverzní funkce.\\

   \begin{figure}[H]
                \hspace{-1cm}
                \includegraphics[scale=0.75]{template-fig/inverz.pdf}
                \centering
            \end{figure}
\end{graph}


\begin{example}
\label{sec: funkce}
\cite{hlinena}
\begin{enumerate}
    \item Nech\v t je $f_2:[0,1] \rightarrow [0,1]$ dána předpisem:
    $$f_2(x)= \begin{cases} \frac x2, & \mbox{pokud~} x \in [0,\frac 12],
    \\ \frac 14, & \mbox{pokud~} x \in ]\frac 12,\frac 34],
    \\ 3.x-2, & \mbox{pokud~} x \in ]\frac 34,1].
    \end{cases}$$
    Funkce $f_2$ není injekce, což znamená, že k ní neexistuje inverzní funkce. Na intervalu $[0,\frac 12]$ je funkce $f_2$ spojitá a injektivní, proto na daném oboru hodnot $[f(0),f(\frac 12)]$ je
    funkce $f_2^{(-1)}$ totožná s~inverzní funkcí
    $f_{2/[f(0),f(\frac 12)]}^{-1}.$
    Potom pro monot\'onní rozšíření inverzní funkce plat\'i
    $$f_2^{(-1)}(x)= \begin{cases} 2x, & \mbox {pokud $x \in [0,\frac 14],$}
    \\ \frac x3 + \frac 23, & \mbox {pokud $x \in ]\frac 14,1].$}
    \end{cases}$$
    Složení funkcí pak bude vypadat následovně:
    $$g(x)=f_2^{(-1)} \circ f_2(x)=f_2^{(-1)} \left(f_2(x)\right)= \begin{cases} \frac 12,
    & \mbox{pokud $x \in [\frac 12,\frac 34],$}
    \\x, & \mbox {jinak.} \end{cases}$$
    
    \item Nech\v t je $f_3:[0,1] \rightarrow [0,1]$ dána předpisem:
    $$f_3(x)= \begin{cases} 2x, & \mbox {pokud $x \in [0,\frac 14],$}
    \\ \frac x3 + \frac 23, & \mbox {pokud $x \in ]\frac 14,1].$}
    \end{cases}$$
    Funkce $f_3$ je injektivní, takže k ní narozdíl od funkce $f_2$ existuje inverzní funkce. Jenomže funkce $f_3$ není bijektivní, tudíž její inverzní funkce nebude definována na celém intervalu $[0,1]$. 
    Její jediné monot\' onní rozšíření na celý interval $[0,1]$ je dáno předpisem
    $$f_3^{(-1)}(x)= \begin{cases} \frac x2, & \mbox {pokud $x \in [0,\frac 12],$}
    \\ \frac 14, & \mbox {pokud $x \in ]\frac 12,\frac 34],$}
    \\ 3.x-2, & \mbox {pokud $x \in ]\frac 34,1]$.}
    \end{cases}$$
\end{enumerate}
\end{example}
Následující grafy zobrazují zmíněné funkce $f_2$ a $f_3$ a jejich složení s inverzními funkcemi.\\

\begin{graph} Zobrazení funkcí z Příkladu \ref{sec: funkce}. Modře jsou vyznačeny (zleva) funkce $f_2$ a $f_3$, zeleně jejich zevšeobecněné inverzní funkce a červeně složené funkce $f_2^{(-1)} \circ f_2(x) $~a~$ f_3^{(-1)}\circ ~f_3(x).$
\label{sec: inverz}
     \begin{figure}[H]
                \hspace{-1cm}
                \includegraphics[scale=0.65]{template-fig/zevs_inverz.pdf}
                \centering
            \end{figure}

            
\end{graph}

Na Obrázku \ref{sec: inverz} lze zřejmě pozorovat následující situaci: $$f_2(x)=f_3^{(-1)}\text{ a } f_3(x)=f_2^{(-1)}(x), \text{ takže } (f_2^{(-1)})^{(-1)}(x)=f_2(x) \text{ a }
            (f_3^{(-1)})^{(-1)}(x)=f_3(x).$$ 
            
 Pro nerostouc\'i funkce je definice a tak\'e konstrukce pseudoinverzn\'i funkce analogick\'a: 
\begin{definition}
    \cite{hlinena}
    Nech\v t je $f:[a,b] \rightarrow [c,d]$ nerostoucí (nekonstantní) funkce, pak pro každé $y \in [c,d]$ je předpisem $$f^{-1}(y) = sup(x \in [a,b];f(x)>y)$$
    definována pseudoinverzní funkce $f^{(-1)}$ k dané funkci f.
\end{definition}


\begin{remark} Na předchozích příkladech bylo naznačeno konstruování pseudoinverzní funkce. Nech\v t je $f:[a,b] \rightarrow [x,d]$ neklesající funkce, lze si pak všimnout určitých poznatk\r u:
    \begin{itemize}
        \item pokud je funkce $f$ bijekce, tak platí $f^{(-1)}(x) = f^{-1}(x),$
        \item  pokud je funkce $f$ rostoucí, její pseudoinverzní funkce je spojitá,
        \item  pro každé $x \in [a,b]$ platí: $f^{(-1)}\circ f(x)\leq x.$
    \end{itemize}
    Nech\v t je $f:[a,b] \rightarrow [x,d]$ nerostoucí funkce:
    \begin{itemize}
        \item pokud je funkce $f$ bijekce, tak platí $f^{(-1)}(x) = f^{-1}(x),$
        \item  pokud je funkce $f$ klesající, její pseudoinverzní funkce je spojitá,
        \item  pro každé $x \in [a,b]$ platí: $f^{(-1)}\circ f(x)\leq x.$
    \end{itemize}
\end{remark}

 Zobecn\v en\'i inverzn\'i funkce n\'am d\'av\'a mo\v znost zobecnit aditivn\'i i multiplikativn\'i generov\'an\'i t-norem:
\begin{sentence} \cite{KMP}
    Nech\v t je $f:[0,1] \rightarrow [0,\infty]$ klesající funkce
    taková, že $f(1)=0$ a pro všechny $(x,y) \in [0,1]^2$  je splněno
    $$f(x)+f(y) \in H(f) \cup [f(0)^+,\infty].$$
    Potom funkce dvou proměnných $T:[0,1]^2 \rightarrow [0,1]$ dána předpisem
    $$T(x,y)=f^{(-1)}(f(x)+f(y))$$
    je triangulární norma.
\end{sentence}
\begin{sentence}\cite{KMP}
    Nech\v t je $l:[0,1] \rightarrow [0,1]$ rostoucí funkce
    taková, že $l(1)=1$ a pro všechny $(x,y) \in [0,1]^2$  je splněno
    $$l(x).l(y) \in H(l) \cup [0,l(0)^+].$$
    Potom funkce dvou proměnných $T:[0,1]^2 \rightarrow [0,1],$ dána předpisem
    $$T(x,y)=l^{(-1)}(l(x).l(y)),$$
    je triangulární norma.
\end{sentence}


\subsection{Fuzzy disjunkce} 
Fuzzy disjunkce vyjadřuje, stejně jako fuzzy konjunkce, míru spojení dvou či více prvk\r u. Výsledek fuzzy disjunkce je hodnota $x \in [0,1].$ 
Základní vlastnosti fuzzy disjunkcí jsou:
\begin{enumerate}
    \item \textbf{Soudržnost:}\\
    Pro hodnoty 0 a 1 se fuzzy disjunkce shoduje s klasickou disjunkcí.
    \item \textbf{Kontinuita hodnot:}\\
    Fuzzy disjunkce nepracuje jenom s  ostrými hodnotami „pravda“ a „nepravda“, ale na stupni neurčitosti či specifickým číselném zápisu, který reflektuje stupeň spojení prvk\r u.
   \item \textbf{Monotonie:}\\
    Fuzzy disjunkce je monotonní operace, což znamená, že s klesajícími vstupními hodnotami klesají i výsledky jejich disjunkce.
\end{enumerate}

\begin{definition}
    \cite{Kolo}
    Neklesající zobrazení $D: [0,1]^2 \rightarrow [0,1]$ se nazývá disjunktor, pokud pro libovolné $a, b \in [0,1]$ platí\\ $D(a,b) = 1$ pokud  $a = 1$, nebo  $b = 1,$\\
    $D(0,0) = 0.$
\end{definition}

Nej\v cast\v eji pou\v z\'iv\'any funkce pro modelov\'an\'i fuzzy disjunkce jsou triangul\'arn\'i konormy, kter\'ym je v\v enov\'ana n\'asleduj\'ic\'i podkapitola.

\subsection{Triangul\'arn\'i konormy} 
\label{sec: Triangulární konormy}

Triangulární konormy (zkráceně t-konormy) jsou, podobn\v e jako a t-normy, re\'aln\'e funkce dvou prom\v enn\'ych:
\begin{definition}
    T-konorma je binární operace na intervalu $[0,1],$ t.j. funkce $S: [0,1]^2 \rightarrow [0,1]$ taková, že pro každé $x, y, z \in [0,1]$ jsou splněny následující axiomy:
    \begin{enumerate}
        \item \textbf{Komutativnost: } $S(x,y) = S(y,x),$
        \item \textbf{Asociativita: } $S(x,S(y,z)) = S(S(x,y),z)$,
        \item \textbf{Monotónnost:} pokud $y \leq z$ pak $S(x, y) \leq S(x, z)$,
        \item \textbf{Okrajová podmínka: } $S(x,0) = x.$
    \end{enumerate}
\end{definition}

Triangulární konormy (zkráceně t-konormy) a t-normy jsou du\'aln\'i funkce:

\begin{sentence}
    \cite{Springer}
    Pokud je $T:[0,1]^2\to [0,1]$ t-norma, tak její duální t-konorma $S: [0,1]^2 \rightarrow [0,1]$ je dána předpisem $$S(x,y) = 1 - T(1-x, 1-y).$$
\end{sentence}


Mezi základní triangulární konormy se obdobně (ale komplementárně) jako u t-norem řadí maximová a Łukasiewiczova t-norma, pravděpodobnostní a drastický součet.
\begin{example}
\cite{Springer}
    \begin{enumerate}
    \item \textbf{Maximová t-konorma} $S_M: [0,1]^2 \rightarrow [0,1]$
    $$S_M(x,y) = max(x,y).$$
    \item \textbf{Pravděpodobnostní součet} $S_P: [0,1]^2 \rightarrow [0,1]$
    $$S_P(x,y) = x+y-x.y.$$
    \item \textbf{Łukasiewiczova t-konorma} $S_L: [0,1]^2 \rightarrow [0,1]$
    $$S_L(x,y) = min(x+y,1).$$
    \item \textbf{Drastický součet} $S_D: [0,1]^2 \rightarrow [0,1]$
    $$S_D:(x)=\begin{cases} max(x,y), & \mbox{pokud  }  min(x,y) = 0,\\ 
    1, &  jinak.  \end{cases}$$
\end{enumerate}
\end{example}

\begin{graph} Maximová t-konorma $S_M$, Pravděpodobnostní součet $S_P$, Łukasiewiczova t-konorma $S_L$, Drastický součet $S_D$.\\

   \begin{figure}[H]
                \hspace{-1cm}
                \includegraphics[scale=0.5]{template-fig/konormy.pdf}
                \centering
            \end{figure}

\end{graph}

Je zajímavé sledovat podobnosti v grafické reprezentaci t-norem a t-konorem, kde je zřetelnější komplementárnost těchto operací.

\subsection{Fuzzy implikace} 


Fuzzy implikace patří mezi nejd\r uležitější fuzzy pojmy a díky takovým spojením výrok\r u je možné mnohem jednodušeji popsat lidské vyjadřování matematicky. Implikace se v obecném jazyce vyjadřuje mnoha zp\r usoby, nejtypičtěji pak slovním spojením \textit{když-tak}. Fuzzy implikace je, podobn\v e jako p\v redchoz\'i logick\'e spojky, monot\'onn\'im roz\v s\'i\v ren\'im klasick\'e implikace.
\begin{definition} \label{impl}
    Zobrazení $I: [0,1]^2 \rightarrow [0,1] $ se nazývá implikátor, pokud
    \begin{itemize}
        \item $I$ je nerostoucí ve svojí první souřadnici,
        \item $I$ je neklesající ve svojí druhé souřadnici,
        \item $I(1,0) = 0, I(0,0) =  I(0,1) = I(1,1) = 1.$
    \end{itemize}
\end{definition}

Modelov\'an\'i fuzzy implikace je ale o něco složitější, než modelov\'an\'i dříve zmíněných logických spojek. Tyto spojky ale mohou být využity pro konstrukci fuzzy implikace. Zřejmě jsou následující logické formule tautologicky ekvivalentní: $$ p\implies q, \mbox{   } \neg p \vee q, \mbox{   } \neg p\vee (p\wedge q) .$$ Lze tedy spojit informace o modelov\'an\'i p\v redchozích logických spojek a jejich vzájemných ekvivalentních \' upravách pro tvorbu fuzzy implikátoru. Konstrukce konjunkce lze vyjádřit t-normou, diskunkce t-konormou a negace standardn\'im negátorem. Vzniknou pak n\'asleduj\'ic\'i p\v redpisy:
$$I(x,y)=S(N(x),y)\text{ neboli (S,N)-implikátory},$$
$$I(x,y)=S(N(x),T(x,y)) \text{ neboli Q-implikátory}.$$

\begin{definition}
    \cite{Springer}
    Funkce $I: [0,1]^2 \rightarrow [0,1]$ se nazývá (S,N)-implikace, pokud existuje t-konorma S a fuzzy negace N taková, že $$I(x,y) = S(N(x)y), x \in [0,1].$$
\end{definition}

\begin{example}
\cite{Springer}
Pro základní t-konormy zmíněné v kapitole \ref{sec: Triangulární konormy} a standardní negátor $N_S$ existují následující implikátory, které patří do skupiny $(S,N)-$implikátor\r u:\\
    \vbox{$$ I_{S_M}(x,y)=\max(1-x,y),$$ }
\vbox{$$ I_{S_P}(x,y)=1-x+x.y,$$}
 \vbox{$$ I_{S_L}(x,y)=\min(1-x+y,1),$$}
 $$ I_{S_D}(x,y)=\begin{cases} 1-x,
&\mbox {pokud y=0,} \\y, &\mbox {pokud x=1}, \\
1, &\mbox {jinak.} \end{cases} $$
\end{example}

Pro představu jsou níže představeny grafy implikátorů $I_{S_M}, I_{S_P}, I_{S_L}$ a $I_{S_D}.$


\begin{graph} Uk\' azka implik\' ator\r u.
    \begin{figure}[H]
                \hspace{-1cm}
                \includegraphics[scale=0.65]{template-fig/impl1.pdf}
                \centering
            \end{figure}
            \begin{figure}[H]
                \hspace{-1cm}
                \includegraphics[scale=0.65]{template-fig/impl2.pdf}
                \centering
            \end{figure}

\end{graph}
Na tomto obrázku je možné pozorovat nápadnou podobnost s grafy t-konorem. Funkce jsou otočeny o 90 ° doleva, což jen dokazuje použití klasick\' eho negátoru ku základním t-konormám. 


Dále je nutné popsat již dříve zmíněné Q-implikátory.
\begin{definition}
    \cite{Springer}
    Funkce $I: [0,1]^2 \rightarrow [0,1]$ se nazývá Q-implikace, pokud existuje t-norma T, t-konorma S a fuzzy negace N taková, že $$I(x,y) = S(N(x),T(x,y)), x \in [0,1].$$
\end{definition}
Mezi nejznámější Q-implikátor patří tzv. Zadeh\r uv implikátor, který staví na minimové t-normě $T_M,$ maximové t-konormě $S_M$ a Zadehově negaci $N_S$. Jeho předpis pak vypadá následovně $$I_Z(x,y) = max(1-x, min(x,y)).$$ Podobně pak lze tvořit i další implikátory, které budou stavět na předchozích negátorech, t-normách a t-konormách. Implikátor $I_P$ je tvořen součinovou t-normou $T_P,$ pravděpodobnostním součtem $S_P,$ negací $N_S$ a má předpis $$I_P(x,y) = 1-x+x^2y.$$  Po použití Łukasiewiczovy t-normy $T_L,$ Łukasiewiczovy t-konormy $S_L,$ a negátoru $N_S$ bude výsledkem klasický S-implikátor s předpisem $$I_L(x,y) = max(1-x, y).$$ V neposlední řadě stojí za zmínku implikátor $I_D,$ který sestává z drastického součinu $T_D,$ drastického součtu $S_D,$ negátoru $N_S$ a vypadá následovně $$I_D(x,y) = \begin{cases}  y, & \mbox{pokud } x = 1,\\
                1 - x, &  jinak.  \end{cases}$$\\ 
Podobným zp\r usobem je samozřejmě možné tvořit nekonečné množství implikací s r\r uznými předpisy, které se budou měnit na základě kombinací t-norem, t-konorem a negátor\r u. Pro lepší pochopení jsou níže vykresleny grafy uvedených implikací.
\begin{graph}Implikátory $I_Z, I_P, I_L, I_D.$
\begin{figure}[H]
                \hspace{-1cm}
                \includegraphics[scale=0.5]{template-fig/impl3.pdf}
                \centering
            \end{figure}
    
\end{graph}

Mezi nejpoužívanější rozšíření klasické implikace na interval $[0,1]$ je \textit{reziduální operátor} $R_T$ pro danou zleva-spojitou t-normu $T$, také označován jako $R-$implikátor: $$ R_T(x,y)=\sup(z \in [0,1]; T(x,z) \leq y).$$

V následujícím příkladu jsou rezidu\'aln\'i implik\'atry odvozen\'e ze zn\'am\'ych zleva spojit\'ych t-norem:
\begin{example} Pro zleva spojité t-normy $T_M, T_P$ a $T_L$ pak dostáváme  následující reziduální implikátory:
    \cite{Springer}\\
     $$ R_{T_M}(x,y)=\begin{cases} 1, &\mbox {pokud $x\leq y$,} \\y, &\mbox{jinak,} \end{cases} $$
    (Göddel\r uv implik\' ator)
     $$ R_{T_P}(x,y)=\min \left (\frac yx,1 \right ),$$
    (Goguen\r uv implikátor)
     $$ R_{T_L}(x,y)=\min(1-x+y,1).$$
    (\L{}ukasiewicz\r uv implikátor)\\
\end{example}
\begin{graph}
     Grafické znázornění R-implikátor\r u.\\
\begin{figure}[H]
                \hspace{-1cm}
                \includegraphics[scale=0.7]{template-fig/impl4.pdf}
                \centering
            \end{figure}
     

\end{graph}



Z uveden\'eho je z\v rejm\'e, \v ze fuzzy implikace jsou funkce dvou prom\v enn\'ych.
V klasick\'e logice m\'a implikace r\r uzn\'e vlastnosti, kter\'e nemus\'i tyto funkce nutn\v e spl\v novat, p\v ritom spl\v nuj\'i v\v sechny podm\'inky z Definice \ref{impl}.

\begin{definition}
\cite{Springer}
Říká se, že fuzzy implik\'ator $I:[0,1]^2 \rightarrow [0,1]$ spl\v nuje:
\begin{enumerate}
\item[(NP)] levý princip neutrality, pokud
$$I(1,y)=y; ~~~~y \in [0,1],$$
\item[(EP)] princip z\'aměny, pokud
$$I(x,I(y,z))=I(y,I(x,z)) \mbox{  pro ka\v zd\'e   } x,y,z \in [0,1],$$
\item[(IP)] princip identity, pokud
$$I(x,x) = 1; ~~~ x \in [0,1], $$
\item[(OP)] vlastnost uspořádání
$$x \leq y \iff I(x,y) =1; ~~~ x,y \in [0,1],$$
\item[(CP)] kontrapozitivitu vzhledem na dan\'y neg\'ator $N$, pokud
$$ I(x,y)=I(N(y),N(x)); ~~~ x,y \in [0,1].$$
 {\item[(LI)] z\'akon přenosu  vzhledem na  t-normu $T$, pokud
$$I(T(x, y), z) = I(x, I(y, z)); ~~~ x,y,z \in  [0, 1].$$
\item[(WLI)]  slab\'y z\'akon přenosu vzhledem na komutativní a
rostoucí funkci $F:
[0,1]^2 \to [0, 1]$, pokud
$$I(F(x, y), z) = I(x, I(y, z)); ~~~  x,y,z \in  [0, 1].$$}
\end{enumerate}
\end{definition}

\begin{remark}
    Bylo dok\'az\'ano, \v ze nap\v r. (S,N)-implikátory spl\v nují v\v zdy $(NP)$ a $(EP)$, reziduální implikátor zase spl\v nuje v\v zdy $(NP)$ a $(OP).$
\end{remark}

\subsection{Konstrukce fuzzy implikátor\r u}

Většina implikátor\r u je ve fuzzy logice modelována pomocí t-norem. Konstrukce nových t-norem, tedy nových konjunktor\r u, je naznačena v předchozích kapitolách. Při konstruování implikátor\r u lze použít podobný postup.

Generování implikátor\r u lze rozdělit do tří tříd. Yager popsal první dvě třídy, tedy $f-$generované a $g-$generované implikátory. Za popsání  $h-$generovaných implikátor\r u, tedy poslední třídy, si připisuje zásluhy Balasubramaniam Jayram. Následně jsou všechny třídy představeny.

\begin{sentence}\cite{yager} 
Pokud je $f: [0,1] \to [0,\infty]$ klesající a spojitá funkce,
přičemž $f(1) = 0,$ potom je funkce $I: [0,1]^2 \to [0,1],$ dána předpisem
$$I(x,y) = f^{-1}(x \cdot f(y)), \mbox {   } x, y \in [0,1],$$
fuzzy implikátor (pričemž $0 \cdot \infty = 0$). \\
\end{sentence}

\begin{example}
    \cite{Springer}
    Konstrukce f-generovaných implikátor\r u m\r uže vypadat následovně.
    Pokud by byl použit aditivní generátor součinové t-normy $T_P,$ tedy 
    pokud $f(x) = - \log x,$ výsledkem bude Yager\r uv implikátor:
    $$I_{YG}(x,y)= \begin{cases} 1,
    &\mbox {pokud $x=0$ a $y=0,$} \\
    y^x, &\mbox {jinak.}
    \end{cases}$$
\end{example}
\begin{remark}
    V předchozím příkladu se nejednalo o $(S,N)-$implikátor ani o R-implikátor. To však neplatí pro všechny f-generované implikátory.
\end{remark}


\begin{sentence}\cite{yager} 
Pokud je $g: [0,1] \to [0,\infty]$ klesající a spojitá funkce,
přičemž $g(1) = 0,$ potom je funkce $I: [0,1]^2 \to [0,1],$ dána předpisem
$$I(x,y) = g^{(-1)}\left (\frac 1x \cdot g(y)\right ), \mbox {   } x, y \in [0,1]$$
fuzzy implikátor (pričemž $\frac{1}{0} = \infty, 0 \cdot \infty = 0$). \\
\end{sentence}
\begin{example}
    \cite{Springer}
    Pokud by byl použit aditivní generátor pravděpodobnostního součtu $S_P,$ tedy 
    pokud $g(x) = - \log (1- x),$ dostaneme následující fuzzy implikátor:
    $$I_{YG}(x,y)= \begin{cases} 1,
    &\mbox {pokud $x=0$ a $y=0,$} \\
    1-(1-y)^x, &\mbox {jinak.}
        \end{cases}$$
\end{example}
\begin{remark}
    Opět se v předchozím příkladu se nejednalo o $(S,N)-$implikátor ani o R-implikátor. To však také neplatí v každém případě g-generovaných implikátor\r u.
\end{remark}

\begin{sentence}\cite{Springer} 
Pokud je $h: [0,1] \to [0,\infty]$ klesající a spojitá funkce,
přičemž $h(1) = 0,$ potom je funkce $I: [0,1]^2 \to [0,1],$ dána předpisem
$$I(x,y) = h^{(-1)}(x \cdot h(y)), \mbox {   } x, y \in [0,1]$$
fuzzy implikátor. \\
\end{sentence}
\begin{example}
    Pokud by byl jako  $h-$generátor funkce $h_n(x) = 1- \frac {x^n}{n}, n
    \in N,$ výsledkem budou následující fuzzy implikátory:
    $$I_n(x,y) = \min \left ( (n - n \cdot x + x \cdot y^n)^{\frac 1n}, 1\right
    ),$$
    kter\'e jsou $(S,N)-$implikátory.
\end{example}
\begin{remark}
    Pro h-negátory platí, že $I_{h_1} = I_{h_2}$ právě když $h_1 = h_2.$
\end{remark}

Všechny třídy generátor\r u jsou, stejně jako v případě spojitých aditivních generátor\r u, jednoznačně dané s proměnlivou kladnou konstantou.

Krom\v e uveden\'ych zp\r usob\r u byly pops\'any i dal\v s\'i konstrukce implik\'ator\r u, n\v ekter\'e zde tak\'e uv\'ad\'ime.

\begin{sentence}
\cite{Springer}
    Nech\v t je  $f:[0,1]\rightarrow [0,\infty]$ klesající funkce taková, \v ze $f(1)=0$. 
Potom funkce $I_f(x,y):[0,1]^2\rightarrow [0,1]$ dan\'a předpisem
 $$I_f(x,y)=\begin{cases} 1 & \mbox{pokud $x \leq y$},\\
f^{(-1)}(f(y^+)-f(x)) & \mbox{jinak,} \end{cases}$$
kde $f(y^+)= \lim\limits_{y \to y^+}f(y)$ a $f(1^+)=f(1)$
je fuzzy implik\'ator. \\
\end{sentence}

Následuje ilustrační příklad třídy generovaných implikátor\r u.\\
\begin{example}
    Nech\v t je $f_1:[0,1] \rightarrow [0,
\infty]$ funkce daná předpisem:
 $$f_1(x)= -\ln(x).$$

Je d\r uležité zmínit, že je funkce klesající. Pro $f_1^{(-1)}$ pak platí:
 $$f_1^{(-1)}(x)=\min \{ e^{-x},1 \}.$$
Pro  funkci $f_1$ pak lze dostat následující implik\'ator:
$$I_{f_1}(x,y)= \begin{cases} 1,    &\mbox{pokud $x \leq y$}, \\
  \frac{y}{x},   & \mbox{jinak.} \end{cases}$$
\end{example}


Dalším typem generovaných implikátor\r u jsou $I^g$ implikátory. Narozdíl od $I_f$ implikátor\r u jsou ale generovány rostoucími funkcemi.

\begin{sentence} \cite{smutna}
    Nech\v t je  $g:[0,1]\rightarrow [0,\infty]$ rostoucí funkce takov\'a, \v ze $g(0)=0$. 
    Potom funkce $I^g(x,y):[0,1]^2 \rightarrow [0,1]$ dan\'a předpisem
    %\begin{equation}\label{g}
    $$I^g(x,y)=g^{(-1)}(g(1-x)+g(y)),$$
    %\end{equation}
    je fuzzy implik\'ator.\\
\end{sentence}

\begin{example}
\cite{Springer}
    Nech\v t jsou $g_1, g_2:[0,1] \rightarrow [0,\infty]$ 
    dan\'e předpisy:
    \begin{itemize}
    \item $g_1(x)=\begin{cases}  x,    &\mbox{pokud $x \leq 0.5$}, \\
    0.5+0.5x,      &\text{jinak,} \end{cases}$
    \item $g_2(x)=-\ln(1-x).$
    \end{itemize}
    Obě dvě funkce $g_1$ a $g_2$ josu rostoucí.
    Pro jejich pseudo-inverzn\'i funkce  $g_1^{(-1)}$ a $g_2^{(-1)}$ plat\'i:
    \begin{itemize}
    \item $g_1^{(-1)}(x)=\begin{cases} x, &\mbox{pokud $x \leq 0,5$}, \\
    0,5,   &\mbox{pokud $0,5 < x \leq 0,75$}, \\
    2x-1,   &\mbox{pokud $0,75 < x \leq 1$}, \\
    1,   &\mbox{pokud $1 < x$}, \end{cases}$
    \item $g_2^{(-1)}(x)=1-e^{-x} ~ \mbox{pro $x \in [0, \infty]$}.$
    \end{itemize}
    Potom se pro funkce $g_1$ a $g_2$ modelují nasleduj\'icí implik\'atory:
    \begin{itemize}
    \item $I^{g_1}(x,y)=\begin{cases}  1-x+y,      &\mbox{pokud $x \geq 0.5, y \leq 0.5, x-y \geq 0.5$}, \\
    0.5,      &\mbox{pokud $x \geq 0.5, y \leq 0.5, 0.25 \leq x-y < 0.5$}, \\
    1-2x+2y,      &\mbox{pokud $x \geq 0.5, y \leq 0.5, x-y < 0.25$}, \\
    \min(1-x+2y,1),       &\mbox{pokud $x < 0.5, y \leq 0.5$}, \\
    \min(2-2x+y,1),       &\mbox{pokud $x \geq 0.5, y>0.5$}, \\
    1,       &\mbox{pokud $x < 0.5, y > 0.5,$} \end{cases}$
    \item $I^{g_2}(x,y)=1-e^{\ln(x(1-y))}=1-x+xy.$
    \end{itemize}
\end{example}

\begin{sentence} \cite{habilitace}
    Nech\v t je $c$ kladn\'a konstanta a $g:[0,1] \to [0,\infty]$
    rostoucí  funkce. Potom implik\'atory $I^g$ a $I^{c \cdot g}$,
    které jsou zalo\v zen\'e na  funkcích $g$ a $c \cdot g$, jsou
    identick\'e.
\end{sentence}
A na z\'av\v er uvedeme zp\r usob konstrukce, s kter\'ym jsme se potkali p\v ri konstrukci t-norem a kter\'emu se budeme v da\v s\'ich kapitol\'ach v\v enovat podrobn\v eji.
\begin{sentence}\cite{Springer}
    Nech\v t $\varphi \in B$, potom pro libovolný implikátor $I: [0,1]^2 \rightarrow [0,1]$ je $\varphi$-transformace $I_\varphi$ také implikátor.
\end{sentence}

\begin{definition}\cite{Springer}
    Nech\v t je $\varphi:[0,1] \rightarrow [0,1]$  rostoucí
    bijekce a množina všech takových bijekcí Nech\v t je B. Nech\v t je funkce
    $I:[0,1]^2\rightarrow [0,1]$ 
    implikátor.
    Pak je funkce
    $$I_\varphi(x,y)=\varphi^{-1}(I(\varphi (x), \varphi (y)))$$
    $\varphi-${\em transformací implikátoru} $I.$ Implikátor se nazývá
    {\em totálně invariantní,} pokud pro libovolnou funkci $\varphi \in B$ platí $I_\varphi=I.$\\
\end{definition}

\section{Konstrukce fuzzy logick\'ych spojek}

V této kapitole se věnuji konstrukci fuzzy logických spojek pomocí funkcí jedné proměnné a již existujících spojek. Dále tyto poznatky využívám při analýze lidského vyjadřování v běžném jazyce a při konstrukci fuzzy logických spojek na něm založených.

\subsection{$\varphi-$transformace triangulárních norem a konorem}

Následující část se věnuje zobecn\v en\'i $\varphi-$transformaci t-norem, konkr\'etn\v e $\varphi-$transformaci $T_M.$ Zobecn\v en\'i bude spo\v c\'ivat v tom, \v ze funkce $\varphi$ nebude rostouc\'i bijekce, bude to jenom neklesaj\'ici funkce, proto zde tak\'e vyu\v zijeme zobecn\v en\'i inverzn\'i funkce. 
\begin{sentence}\cite{KMP}\label{subnorma}
     Nech\v t je $\varphi : [0,1] \rightarrow [0,1]$ neklesající funkce a  spojitá na otev\v ren\'em intervalu
 $]0,1[$ a nech\v t $T: [0,1]^2 \rightarrow [0,1]$ je t-norma. Potom funkce
    $(T)_\varphi : [0,1]^2 \rightarrow [0,1]$ je dána p\v redpisem $$(T)_\varphi (x,y)= \begin{cases} min(x,y), & \mbox {pokud } max(x,y) = 1,
    \\ \varphi^{(-1)}[T(\varphi(x), \varphi(y))], & \mbox {jinak,}
    \end{cases}$$
    je t-norma.
  \end{sentence}
  V této práci se blí\v ze seznámíme s $\varphi-$transformaci $T_M$ a využijeme zobecnění předchozí věty:

\begin{sentence} 
\cite{hlinena}
\label{smut} Nech\v t je $\varphi \colon [0,1] \to [0,1]$ neklesající funkce. Pak je funkce $(T_M)_\varphi\colon[0,1]^2\to[0,1]$  t-normou pouze v případě, že
$$\varphi^{(-1)}\circ\varphi\circ\varphi^{(-1)}\circ\varphi(x)\in \{\varphi^{(-1)}\circ\varphi(x),\varphi^{(-1)}\circ\varphi\circ\varphi^{(-1)}\circ\varphi(1^-)\}$$
pro každé $x\in[0,1],$ kde $\varphi^{(-1)}\circ\varphi\circ\varphi^{(-1)}\circ\varphi(1^-)=\lim\limits_{x\to 1^-}\varphi^{(-1)}\circ\varphi\circ\varphi^{(-1)}\circ\varphi(1^-).$
\end{sentence}
\begin{remark}
    Nech\v t $\varphi\colon[0,1]\to [0,1]$ je neklesající, zleva spojitá  funkce a nechť $a=\inf\{x \in [0,1]\colon \varphi(x)=1\}.$ Potom operace $(T_M)_\varphi\colon[0,1]^2\to[0,1]$ je t-normou právě když platí $\varphi^{(-1)}\circ\varphi\circ\varphi^{(-1)}\circ\varphi(x)=\varphi^{(-1)}\circ\varphi(x)$ pro každé $x \in [0,a].$ Tato  t-norma je zleva spojitá na intervalu  $]0,1[^2$, ale není zleva spojitá na  intervalu $[0,1]^2.$
\end{remark}

Uveden\'e podm\'inky znamenaj\'i, \v ze funkce $\varphi$ nem\r u\v ze m\'it dva konstantn\'i \'useky v bezprost\v redn\'i bl\'izkosti. P\v ri poru\v sen\'i t\'eto podm\'inky by v\'ysledn\'a funkce nebyla asociativn\'i.
\bigskip


  
  N\'asleduj\'ic\'i p\v r\'iklad objas\v nuje situaci pro $T_M$ a jej\'i $\varphi-$transformaci, kdy\v z $\varphi$ je neklesaj\'ic\'i funkce. 
\begin{example}
\label{sub: fi}
Nech\v t $\varphi:[0,1] \to [0,1]$ je d\'ana p\v redpisem:    
 $$\varphi(x) = \begin{cases} x, & \mbox {pokud } x \in [0,\frac{1}{3}],
    \\ \frac{1}{3}, & \mbox {pokud } x \in [\frac{1}{3}, \frac{2}{3}],\\
    2x - 1, & \mbox {pokud } x \in [\frac{2}{3}, 1].
    \end{cases}$$
Potom pro $\varphi-$transformaci $T_M$ plat\'i
$$(T_M)_{\varphi(x)} = \begin{cases} \frac{1}{3}, & \mbox {pokud } (x, y) \in [\frac{1}{3},\frac{2}{3}]^2 \cup [\frac{1}{3}, \frac{2}{3}] \times [\frac{2}{3}, 1] \cup [\frac{2}{3}, 1] \times [\frac{1}{3}, \frac{2}{3}],\\
    T_M, & \mbox {jinak.}
    \end{cases}$$
\end{example}
Pro lep\v s\'i n\'azornost uv\'ad\'im graf  funkce $\varphi$ a jej\'i pseudoinverzn\'i funkce a tak\'e vizu\'aln\'i rozd\v elen\'i grafu funkce $\left(T_M\right)_\varphi.$\\
\begin{graph}
    Vizu\' aln\' i zobrazení funkce $\varphi$ a jej\'i pseudoinverzn\'i funkce
\begin{figure}[H]
                \hspace{-1cm}
                \includegraphics[scale=0.75]{template-fig/phi_transform.pdf}
                \centering
            \end{figure}
\end{graph}

    
\begin{graph}Vizu\' aln\' i rozd\v elen\'i grafu $\varphi$-transformace minimov\' e t-normy\\
\tikzset{>=stealth}
\centering
\includegraphics[scale=0.8]{template-fig/phi-tnorm.pdf}
\end{graph}

%\begin{tikzpicture}
%    \draw[step=1cm,black,thin] (0,0) grid (10,10);
%   \foreach \xtick in {0,...,10} {\pgfmathsetmacro\result{\xtick * .1} \node at (\xtick,-0.5) 
% {\pgfmathprintnumber{\result}}; }
%    \foreach \ytick in {0,...,10} {\pgfmathsetmacro\result{\ytick * .1} \node at (-.5,\ytick) 
%{\pgfmathprintnumber{\result}}; }
    
%\end{tikzpicture} 
Příklad \ref{sub: fi} nám ilustruje vliv průběhu funkce $\varphi$ na průběh t-normy, která vznikla $\varphi-$transformací. Pro obecnou funkci $\varphi$ a t-normu $T_M$ dostáváme funkci $(T_M)_{\varphi}$, předpis které popisuje následující tvrzení.
\begin{sentence}
\cite{smutna}
\label{t-norm}
 Nech\v t je $\varphi:[0,1]\rightarrow [0,1]$
neklesající funkce spojitá na otevřeném intervalu $]0,1[$.
Nech\v t je $\{[a_i,b_i]\}_{i\in I}$ množina podinterval\r u
intervalu $[0,1]$ taková, že $\varphi(x)=c_i$ pro $x\in
[a_i,b_i]$ (přičemž na intervalu $[0,1]\backslash \bigcup \limits_{i \in I}
[a_i,b_i]$ je funkce $\varphi$
rostoucí).
Pak je operátor $(T_M)_{\varphi}$  t-norma daná předpisem
$$ (T_M)_{\varphi}(x,y) = \begin{cases} \varphi^{(-1)}(c_i)=a_i, &\mbox {pokud
$(x,y)\in [a_i,1[\times[a_i,b_i]$ nebo}
\\ & (x,y)\in [a_i,b_i]\times[b_i,1[,
\\ T_M(x,y), &\mbox {jinak.}
\end{cases} $$
\end{sentence}
\begin{example}
    \label{conormphi}
    Vzhledem k tomu, \v ze ke ka\v zd\'e t-norm\v e lze sestrojit jej\'i du\'aln\'i t-konormu, tak aplikujeme na $\left(T_M\right)_\varphi$ vztah $S(x,y)=1-T(1-x,1-y)$ a pro $\left(S_M\right)_\varphi$ dostaneme
 $$(S_M)_\varphi(x,y) = \begin{cases} \frac{2}{3}, & \mbox {pokud } (x,y) \in [0,\frac{1}{3}[\times[\frac{1}{3},\frac{2}{3}] \cup [\frac{1}{3},\frac{2}{3}]\times[0,\frac{1}{3}[ \cup [\frac{1}{3},\frac{2}{3}]^2,
    \\ S_M, & \mbox {jinak. }
    \end{cases}$$
    \begin{graph}Vizu\' aln\' i rozd\v elen\'i grafu $\varphi$-transformace maximové t-konormy
    \label{graph: max-conorm}


\centering
\includegraphics[scale=0.8]{template-fig/phi-t-conorm.pdf}
\end{graph}

\end{example}
Příklad \ref{conormphi} nám ilustruje vliv průběhu funkce $\varphi$ na průběh t-konormy, která vznikla $\varphi-$transformací. Pro obecnou funkci $\varphi$ a t-konormu $S_M$ dostáváme funkci $(S_M)_{\varphi}$, předpis které popisuje následující tvrzení.

\begin{sentence}
    
 Nech\v t je $\varphi:[0,1]\rightarrow [0,1]$
neklesající funkce spojitá na otevřeném intervalu $]0,1[$.
Nech\v t je $\{[a_i,b_i]\}_{i\in I}$ množina podinterval\r u
intervalu $[0,1]$ taková, že $\varphi(x)=c_i$ pro $x\in
[a_i,b_i]$ (přičemž na intervalu $[0,1]\backslash \bigcup \limits_{i \in I}
[a_i,b_i]$ je funkce $\varphi$
rostoucí).
Pak je operátor $(S_M)_{\varphi}$  t-konorma daná předpisem
$$ (S_M)_{\varphi}(x,y) = \begin{cases} 1-\varphi^{(-1)}(c_i), &\mbox {pokud
$(x,y)\in [1-b_i,1-a_i[\times[0,1-a_i]$ nebo}
\\ & (x,y)\in [0,1-b_i]\times[1-b_i,1-a_i[,
\\ S_M(x,y), &\mbox {jinak.}
\end{cases} $$
\end{sentence}



\subsection{Konstrukce $(S,N)-$implik\'ator\r u pomoc\'i $\varphi-$transformace t-konorem}
 V této kapitole se věnuji konstrukcím $(S,N)-$implikací založených na standardní Zadehově negaci a $(S_M)_\varphi$, přičemž $\varphi$ je neklesající zleva spojitá funkce.
 
\begin{example}
    Nech\v t je $\varphi$ funkce z p\v ríkladu \ref{sub: fi}.  Pomocí standardní fuzzy negace $N_S(x)$ a maximové t-konormy $(S_M)_\varphi$  zkonstruujeme $(S,N)-implikaci$. 
\end{example}
   
     \begin{graph} 
     Vizu\' aln\' i rozd\v elen\'i grafu $\varphi$-transformace $(S,N)-$implikace tvořené maximovou t-konormou\\
        \centering
        \includegraphics[scale=0.8]{template-fig/phi-impli.pdf}
    \end{graph}


 Konstrukci tohoto $(S,N)-$implikátoru lze tedy popsat následujícím předpisem:
    Nech\v t je  $\varphi:[0,1]\rightarrow [0,1]$
funkce z příkladu \ref{sub: fi}.
Potom je implikace $(S,N)-$ na základě $N_S$ a $(S_M)_{\varphi}$ dána vzorcem
$$ I_{(S_M)_{\varphi}}(x,y) = \begin{cases} \frac{2}{3}, &\mbox {pokud~~}
(x,y)\in [\frac{1}{3},\frac{2}{3}[\times[0,\frac{2}{3}] \mbox{~~nebo~~}
\\ & (x,y)\in [\frac{2}{3},1]\times[\frac{1}{3},\frac{2}{3}[,
\\ \max(1-x,y), &\mbox {jinak.}
\end{cases} $$

Pro obecnou nekleasjící a zleva spojitou funkci $\varphi$, standardní negátor a maximovou t-konormu dostáváme fuzzy implikátor, kterého obecný předpis je v následujícím tvrzení.
\begin{sentence}
       Nechť $\varphi:[0,1]\rightarrow [0,1]$
je neklesající a zleva spojitá funkce na intervalu $]0,1[$.
Nechť
$\{[a_i,b_i]\}_{i\in I}$ je množina podintervalů intevalu $[0,1]$, přičemž $\varphi(x)=c_i$ for $x\in
[a_i,b_i]$.
Potom $(S,N)-$implikátor založený na $N_S$ and $(S_M)_{\varphi}$ je dán formulou
$$ I_{(S_M)_{\varphi}}(x,y) = \begin{cases} 1-\varphi^{(-1)}(c_i)=1-a_i, &\mbox {~~pokud~~}
(x,y)\in [a_i,b_i[\times[0,b_i] \mbox{~~nebo~~}
\\ & (x,y)\in [b_i,1]\times[a_i,b_i[,
\\ \max(1-x,y), &\mbox {jinak.}
\end{cases} $$

\end{sentence}

 Pokud uvažujeme funkci $\varphi$, jež nespl\v nuje podmínku z Věty \ref{smut}, pak $(S_M)_{\varphi}$ nebude asociativní, ale bude stále fuzzy disjunkcí D. Použitím operace $D(N_s(x),y)$ dostaneme implikaci, která bude mít podobné vlastnosti jako $(S,N)-$implikace. Nespl\v nuje ale princip záměny, jelikož se tato vlastnost váže k asociativitě použité t-konormy. Tímto způsobem získáme novou třídu fuzzy implikací.


\subsection{Konstrukce $R-$implik\'ator\r u pomoc\'i $\varphi-$transformace t-norem}
Tato část je věnována konstrukcím reziduálních implikací založených na $\varphi-$transformaci minimové t-normy. 

{\color{red}Výchozí reziduální implikace jsou založeny na zleva spojitých t-normách, přičemž $\varphi-$transformace minimové t-normy, tedy $(T_M)_\varphi$, není zleva spojitá na jednotkovém intervalu $[0,1]^2$. Z $(T_M)_\varphi$ lze pak jednoduše získat zleva spojitý operátor $C_\varphi:[0,1]^2 \to [0,1]$, který není t-norma, ale vyhovuje všem vlastnostem fuzzy konjunkce. Pro neklesající zleva spojitou funkci $\varphi: [0,1] \to [0,1]$ je daný operátor popsán následovně:}

\begin{equation} \label{rez}
    C_{\varphi}(x,y) = \begin{cases} \varphi^{(-1)}(c_i)=a_i, &\mbox {pokud
$(x,y)\in [a_i,1]\times[a_i,b_i]$ nebo}
\\ & (x,y)\in [a_i,b_i]\times[b_i,1],
\\ T_M(x,y), &\mbox {jinak.}
\end{cases} 
\end{equation} 

\begin{example}
Nech\v t je $\varphi$ funkcí z příkladu \ref{sub: fi} a $C_\varphi$ nech\v t je operátorem daným formulí \ref{rez}. Lze pak zkonstruovat následující reziduální implikaci: $$R_{C_\varphi}(x,y)=\sup\{z \in [0,1]; C_\varphi(x,z) \leq y\}.$$
    
    \begin{graph} Vizu\' aln\' i rozd\v elen\'i grafu $\varphi$-transformace R-implikace tvořené minimovou t-normou\\
        \centering
        \includegraphics[scale=0.8]{template-fig/RPhi-impl.pdf}
    \end{graph}
\end{example}

Konstrukci reziduální implikace postavené n popsána následujícím výrokem:

\begin{sentence}
    {\color{blue} nebo výrok?}
    Nech\v t je $\varphi:[0,1]\rightarrow [0,1]$
funkce z příkladu \ref{sub: fi}.
Pak platí, že operátor $R_{C_{\varphi}}$ je fuzzy implikace a je dán předpisem
$$ R_{C_{\varphi}}(x,y) = \begin{cases} 1, & \mbox{pokud } x\leq y \mbox{ nebo } (x,y) \in \frac{1}{3},\frac{2}{3}]^2\\
\frac{2}{3}, &\mbox {pokud }
(x,y)\in [\frac{2}{3},1[\times[\frac{1}{3},\frac{2}{3}]
\\ y, &\mbox {jinak.}
\end{cases} $$
\end{sentence}


\subsection{$\varphi-$transformace fuzzy implik\'ator\r u} 
{\color{blue}mozna bych tuto cast dala pred transformovane implikace, at prvne vysvetlujeme transform obecne a pak sn a rezidual zvlast? - pokud ano, prehodim to a trochu poupravim, aby to davalo smysl}
V této části se věnuji obecným konstrukcím fuzzy implikátor\r u pomocí $\varphi$-transformací. Vycházím tedy z klasické Zadehovy negace, $\varphi$-transformace minimové t-normy a maximové t-konormy.
Nech\v je tedy $\varphi: [0,1] \to [0,1]$ jakákoliv neklesající zleva spojitá funkce na intervalu ]0,1[.
\begin{graph}
    Příklad obecné funkce $\varphi^{(-1)}$.\\
    \centering
        \includegraphics[scale=0.6]{template-fig/basic.pdf}
\end{graph}
Pomocí $\varphi$-transformace minimové t-normy lze pak získat přepis, který je popsán ve větě \ref{t-norm}, pro výslednou transformovanou t-normu, která bude tvořit na základě funkce $\varphi$ střídající se konstantní hodnoty s minimovou t-normou. Aby byla výsledná t-norma asociativní, musejí se takové \' useky střídat, t-norma pak tedy nabývá graficky obdobného tvaru jako je písmeno \clqq L\crqq.
\begin{graph}
    Příklad obecné $\varphi$-transformace minimové t-normy.
    \begin{figure}[H]
                \hspace{-1cm}
                \includegraphics[scale=0.6]{template-fig/t-norma.pdf}
                \centering
            \end{figure}

\end{graph}

Jelikož jsou t-konormy funkce duální k t-normám, jejich $\varphi$-transformace jsou založeny na stejném principu. Obecný přepis pro transformovanou t-konormu uvádím ve větě \ref{t-conorm}.
\begin{graph}
    Příklad obecné $\varphi$-transformace minimové t-konormy.
    \begin{figure}[H]
                \hspace{-1cm}
                \includegraphics[scale=0.6]{template-fig/t-konorma.pdf}
                \centering
            \end{figure}
\end{graph}

Obecně lze pak popsat konstrukci $(S,N)-$implikací pomocí $N(x)$ a $\left(S_M\right)_\varphi$, kde je $\varphi$ libovolná neklesající zleva spojitá funkce, následujícím výrokem:
\begin{sentence}
{\color{blue} nebo výrok?}
    Nech\v t je  $\varphi:[0,1]\rightarrow [0,1]$
neklesající funkce, která je zleva spojitá na intervalu $]0,1[$.
Nech\v t je $\{[a_i,b_i]\}_{i\in I}$ množinou podintervalů jednotkového intervalu $[0,1]$ takovou, že $\varphi(x)=c_i$ pro $x\in
[a_i,b_i]$.
Potom je implikace $(S,N)-$ na základě $N_S$ a $(S_M)_{\varphi}$ dána vzorcem
$$ I_{(S_M)_{\varphi}}(x,y) = \begin{cases} 1-\varphi^{(-1)}(c_i)=1-a_i, &\mbox {pokud
$(x,y)\in [a_i,b_i[\times[0,b_i]$ nebo}
\\ & (x,y)\in [b_i,1]\times[a_i,b_i[,
\\ \max(N(x),y), &\mbox {jinak.}
\end{cases} $$
\end{sentence}

\begin{corollary}
    Jestliže existuje funkce $\varphi$, jež nespl\v nuje podmínku z Věty \ref{smut}, pak nastane situace taková, že operátor $(T_M)_\varphi$ není asociativní, a tedy není triangulární norma. Nicméně je ale stále fuzzy konjunkce. Obdobně lze uvažovat i u jeho duálního operátoru $(S_M)_\varphi$.
\end{corollary}

\begin{graph}
    Příklad grafu $\varphi$-transformace (S,N)-implikace tvořené maximovou t-konormou.
    \begin{figure}[H]
                \hspace{-1cm}
                \includegraphics[scale=0.6]{template-fig/implikace.pdf}
                \centering
            \end{figure}
\end{graph}

Reziduální implikace se tvoří o něco složitějším zp\r usobem. Obecně lze tedy popsat konstrukci reziduálních implikací pomocí $\left(T_M\right)_\varphi$, kde je $\varphi$ libovolná neklesající zleva spojitá funkce, následujícím výrokem:

\begin{sentence}
    {\color{blue} nebo výrok?}
    Nech\v t je $\varphi:[0,1]\rightarrow [0,1]$
neklesající funkce, která je zleva spojitá na jednotkovém intervalu $]0,1[$.
Nech\v t je $\{[a_i,b_i]\}_{i\in I}$ skupina podinterval\r u jednotkového intervalu $[0,1]$ taková, že $\varphi(x)=c_i$ pro $x\in
[a_i,b_i]$.
Pak platí, že operátor $R_{C_{\varphi}}$ je fuzzy implikace a je dán předpisem
$$ R_{C_{\varphi}}(x,y) = \begin{cases} 1, & \mbox{pokud } x\leq y \mbox{ nebo } (x,y) \in [a_i,b_i]^2\\
1-\varphi^{(-1)}(c_i)=1-a_i, &\mbox {pokud }
(x,y)\in [b_i,1[\times[a_i,b_i]
\\ y, &\mbox {jinak.}
\end{cases} $$
\end{sentence}

\begin{corollary}
    Konstrukcí popsánou výše byla získána nová třída fuzzy implikací. Na rozdíl od reziduálních implikací založených na zleva spojitých t-normách, tato nová třída nespl\v nuje vlastnost uspořádání.
\end{corollary}


\chapter{Experiment modelování fuzzy implikace}
V rámci své bakalářské práce provádím experiment, který využívá znalostí z předchozích kapitol a zaměřuje se na zjištění toho, zda lidé v běžném jazyce chápou implikace obdobně, jako je popisuje fuzzy matematika nebo nikoliv. Snažila jsem se tedy najít implikaci, která bude co nejvíce odpovídat získané funkci ze zpracovaných dat a vlastně celkově zkoumat, zda lidé nerozumí implikacím spíše jako výrok\r um sobě ekvivalentním.

Data, která využívám pro provádění experimentu jsem získala pomocí mnou vytvořené dotazníkové webové aplikaci\footnote{Dotazník je dostupný pod odkazem: \href{https://www.stud.fit.vutbr.cz/~xjirmu00/bp/}{Dotazník}}. Dotazovaní měli za \' ukol přiřadit k předem připraveným výrok\r um nějakou číselnou hodnotu v rozmezí 1-10, přičemž 1 znamenalo, že s výrokem nesouhlasí a 10, že s ním plně souhlasí. Výrok\r u je celkově dvacet, přičemž se dělí na pět skupin po čtyřech výrocích, kde jsou první dva výroky $A$ a $B$ prosté, a další dva jsou tvořené implikací $A \to B$ a $B \to A$.

Respondenty jsem pečlivě vybírala velmi rozmanitě, protože je zřejmé, že technicky založené osoby vnímají výrokovou logiku více matematicky než-li ostatní lidé. Zvolila jsem tedy rozsáhlou škálu osob a na dotazník odpovídali jak studenti z informatických fakult, tak pracující z r\r uzných obor\r u netýkajících se informačních technologií, ale i d\r uchodci a další.

Použité výroky v dotazníku vypadaly následovně:
\begin{enumerate}
       \item  "Testy z matematiky bývají zpravidla těžké.", 
       \item "Výsledky z testu z matematiky jsou obvykle špatné.",
       \item "Pokud je test z matematiky těžký, mám z něj špatný výsledek.",
       \item  "Pokud mám z testu z matematiky špatný výsledek, test byl těžký.",
       \item  "Chodím dříve spát.",
       \item  "Nebývám příliš unavený.",
       \item  "Když půjdu brzy spát, budu další den méně unavený.",
       \item  "Jsem-li méně unavený, šel jsem předchozí noc dříve spát.",
       \item  "Jízdné je drahé.",
       \item  "Jízda hromadnou dopravou je pohodlná.",
       \item  "Jestliže je jízdné drahé, cesta hromadnou dopravou je pak pohodlná.",
       \item  "Jestliže je cesta hromadnou dopravou pohodlná, jízdné je drahé.",
       \item  "Jsem poměrně vysoký.",
       \item  "Nosím větší velikost bot.",
       \item  "Jsem-li vysoký, nosím pak obvykle větší velikost bot.",
       \item  "Pokud nosím větší velikost bot, jsem pak obvykle vysoký.",
       \item  "Jsem starý.",
       \item  "Mám hodně zkušeností.",
       \item  "Pokud jsem starý, mám hodně zkušeností.",
       \item  "Mám-li hodně zkušeností, jsem starý."
\end{enumerate}
\section{Vyhodnocení experimentu}
\chapter{Závěr}